\documentclass[10pt,a4paper, ngerman]{beamer}

\include{beamer}

\AtBeginSection{\frame{\frametitle{Gliederung}\tableofcontents[currentsection]}}

\setbeamercovered{transparent}
\author{Luca Kiebel}
\title{Kernfusion und Fusionsreaktoren}
%\subtitle{subtitle}
\date{21. Februar 2018}
\institute[HBBK]{Hans-Böckler-Berufskolleg}
\setlength{\itemsep}{10pt}
\begin{document}
\begin{frame}
\titlepage
\end{frame}

\section[Energie aus Masse]{Äquivalenz von Masse und Energie}
\begin{frame}{\secname}{\subsecname}
\begin{itemize}[<+->]
\item hohe Effizienz --> riesige Mengen Energie
\item \(E=mc^2\)
\item E --> Ruheenergie (ohne Bewegung)
\end{itemize}
\end{frame}

\begin{frame}{\secname}{\(E=mc^2\) vs. \(E_{kin}=mgh\)}
\begin{exampleblock}{Beispiel 1: Kugelschreiber im freien Fall}
	\textasciitilde14g Gewicht \\
	0.22 J
\end{exampleblock}
\pause
\begin{exampleblock}{Beispiel 2: Little Boy} %6. August 1945 Hiroshima
	\alt<3,4>{> 70 Kugelschreiber}{< 1 kg} genzündetes Reaktionsmaterial\ftn{1}{https://de.wikipedia.org/wiki/Little\_Boy} \\
	\(5.4*10^{16}\) J 
\end{exampleblock}
\pause[4]
\begin{exampleblock}{Beispiel 3: Katze}
	\textasciitilde5kg Gewicht \\
	\textasciitilde\(4.2*10^{17}\) J\ftn{2}{https://youtu.be/t-O-Qdh7VvQ?t=10}
\end{exampleblock}
\end{frame}


\section{Kernfusion}
\subsection{Geschichte der Forschung}
\begin{frame}{\subsecname}{\secname}
\begin{enumerate}
\item 1917: Erste Kernreaktion (Rutherford)\ftn{1}{http://web.lemoyne.edu/{\textasciitilde}giunta/rutherford.html}

\item 1920: Fusionsreaktion mögliche Energiequelle von Sternen\ftn{2}{Hans Bethe: \textit{Energy Production in Stars}{,} Phys. Rev. 55{,} 1939{,} S. 434–456}

\item 1934: Erste Fusionsreaktion im Labor\ftn{3}{M.L.E. Oliphant{,} Lord Rutherford: \textit{Transmutation effects Observed with Heavy Hydrogen}{,} Rev. 144{,} 1934{,} S. 692}

\item ab 1945: Erforschung der Nutzung von Fusionsreatkionen in Atombomben

\item 1952: Zündung der ersten Wasserstoffbombe\ftn{4}{http://nuclearweaponarchive.org/Usa/Tests/Ivy.html}

\item 1991: Erste kontrollierte Kernfusion zur Energiegewinnung\ftn{5}{P-H Rebut: \textit{The JET preliminary tritium experiment}{,} Rev. 34{,} 1992}
\end{enumerate}
\end{frame}

\subsection{Bedingungen für die Fusion}
\begin{frame}{\subsecname}{\secname}
\pftn{https://science.howstuffworks.com/fusion-reactor2.htm}
\begin{itemize}
\item Hitze: 100 Millionen Kelvin
\item --> Wasserstoff ist Plasma
\item Druck: Atomkerne \textasciitilde1 Femtometer entfernt
\item Sonne: Gravitation; Erde: Magnete
\end{itemize}
\end{frame}


\subsection{Fusion in der Sonne}
\begin{frame}{\subsecname}{\secname}
\begin{itemize}
\item Licht und Wärme der Sonne entstehen in Fusionsreaktionen \pause
\item Genauer: Proton-Proton-Reaktion:
\end{itemize}
\end{frame}

\begin{frame}[fragile]{\subsecname}{\secname}
\begin{figure}
\pftn{https://de.wikipedia.org/wiki/Datei:FusionintheSun.svg}
\centering
\includegraphics[height=0.8\textheight]{fusion-in-der-sonne}
\caption{Fusion in der Sonne}
\label{fig:fusion-in-der-sonne2}
\end{figure}
\end{frame}

\begin{frame}{\subsecname}{\secname}
\pftn{https://de.wikipedia.org/wiki/Proton-Proton-Reaktion}
\begin{itemize}
\item \({\displaystyle \mathrm {{}^{1}H+{}^{1}H\to {}^{2}H+e^{+}+\nu _{e}+0{,}42\;MeV} }\)
\end{itemize}
\centering
\includegraphics[height=0.4\textheight]{fids1}
\end{frame}

\begin{frame}{\subsecname}{\secname}
\pftn{https://de.wikipedia.org/wiki/Proton-Proton-Reaktion}
\begin{itemize}
\item \({\displaystyle \mathrm {{}^{2}H+{}^{1}H\to {}^{3}He+\gamma +5{,}493\;MeV} }\)
\end{itemize}
\centering
\includegraphics[height=0.5\textheight]{fids2}
\end{frame}

\begin{frame}[fragile]{\subsecname}{\secname}
\begin{figure}
\pftn{https://de.wikipedia.org/wiki/Datei:FusionintheSun.svg}
\centering
\includegraphics[height=0.8\textheight]{fusion-in-der-sonne}
\caption{Fusion in der Sonne}
\label{fig:fusion-in-der-sonne}
\end{figure}
\end{frame}

\begin{frame}{\subsecname}{\secname}
\pftn{https://de.wikipedia.org/wiki/Proton-Proton-Reaktion}
\begin{itemize}
\item Proton-Proton-I-Kette: 83,30 \%
\item \({\displaystyle \mathrm {{}^{3}He+{}^{3}He\to {}^{4}He+2\,{}^{1}H+12{,}86\;MeV} }\)
\end{itemize}
\centering
\includegraphics[height=0.5\textheight]{fids3}
\end{frame}

\begin{frame}{\subsecname}{\secname}
\pftn{https://de.wikipedia.org/wiki/Proton-Proton-Reaktion}
\begin{itemize}
\item Proton-Proton-I-Kette: 83,30 \%
\item Proton-Proton-II-Kette: 16,68 \%
\begin{itemize}
 \item 2x \({\displaystyle \mathrm {{}^{4}He}}\)
\end{itemize}
\item Proton-Proton-III-Kette: 0,02 \%
\begin{itemize}
	\item 2x \({\displaystyle \mathrm {{}^{4}He}}\)
\end{itemize}
\end{itemize}
\end{frame}

\begin{frame}{\subsecname}{\secname}
\begin{itemize}
\pftn{http://www.bbc.co.uk/schools/gcsebitesize/science/add\_aqa/stars/lifecyclestarsrev3.shtml}
\item Meiste Fusion findet im Kern statt
\item Größter Teil des Lebens eines Sternes: H --> He
\item Nachdem der H-Vorrat ausgeht: Schwerere Elemente, bis Fe
\item Schwerer als Fe: Supernova
\end{itemize}
\end{frame}


\subsection{Wasserstoffbombe}
\begin{frame}{\subsecname}{\secname}
\pftn{https://www.focus.de/7546710}
\LARGE Atombombe: \normalsize
\begin{itemize}
\item Sprengstoff verdichtet Spaltmaterial
\item --> Kettenreaktion wird ausgelöst 
\end{itemize}
\pause
\vspace{1cm}
\LARGE Wasserstoffbombe: \normalsize
\begin{itemize}
\item Atombombe verdichtet Spaltmaterial
\item --> Fusionsreaktionen beginnen
\end{itemize}
\end{frame}

\begin{frame}{\subsecname}{\secname}
\pftn{https://de.wikipedia.org/wiki/Datei:Teller-ulam-multilang.svg}
\begin{figure}
\centering
\includegraphics[height=0.8\textheight]{h-bomb}
\caption{Wasserstoffbombe}
\label{fig:h-bomb}
\end{figure}
\end{frame}

\begin{frame}{\subsecname}{\secname}
\pftn{https://de.wikipedia.org/wiki/Datei:BombH\_explosion.svg}
\begin{figure}
\centering
\includegraphics[width=\linewidth]{h-bomb-expl}
\caption{Explosion einer Wasserstoffbombe}
\label{fig:h-bomb-expl}
\end{figure}
\end{frame}


\section{Fusionsreaktor}
\subsection[Magnetische Fusion]{Magnetische Fusion (Tokamak / Stellarator)}
\begin{frame}{Magnetische Fusion}{\secname}
\pftn{https://de.wikipedia.org/wiki/Fusion\_mittels\_magnetischen\_Einschlusses}
\begin{itemize}
\item Meistverfolgter Entwicklungsweg für Gewinnung von Energie aus Kernfusion
\item Fortgeschrittener und vielversprechender als Trägheitsfusion
\pause
\item Plasma kann nicht in herkömmlichen Gefäßen gehalten werden
\item --> Einschluss in B-Feld
\end{itemize}
\end{frame}

\begin{frame}{Tokamak}{\secname}
\begin{figure}
\pftn{https://www.euro-fusion.org/tokamak-principle-2}
\centering
\includegraphics[width=0.75\linewidth]{tokamak1}
\caption{Tokamak-Prinzip}
\label{fig:tokamak1}
\end{figure}
\end{frame}

\begin{frame}{Tokamak - Funktion}{\secname}
\pftn{https://www.quora.com/How-does-a-tokamak-work}
\begin{enumerate}
\item Plasma wird durch Hitze und Druck erzeugt
\item Plasma wird durch Magnete in Torusform gehalten
\item Im Plasma fusioniert Wasserstoff zu Proton-Proton-II-Kette
\end{enumerate}
\end{frame}

\begin{frame}{Stellarator}{\secname}
\begin{figure}
\pftn{https://en.wikipedia.org/wiki/Stellarator}
\centering
\includegraphics[width=0.75\linewidth]{stellarator1}
\caption{Stellarator-Prinzip}
\label{fig:stellarator1}
\end{figure}
\end{frame}

\begin{frame}{Stellarator - Funktion}{\secname}
\pftn{https://en.wikipedia.org/wiki/Stellarator}
Funktionsweise ähnelt der eines Tokamak, Form des Plasmas unterscheidet sich.
\end{frame}

\subsection{Trägheitsfusion}
\begin{frame}{\subsecname}{\secname}
\pftn{https://de.wikipedia.org/wiki/Datei:Inertial\_confinement\_fusion.svg}
\begin{figure}
\centering
\includegraphics[width=\linewidth]{traegheitsfusion}
\caption{Zünden einer Trägheitsfusionsreaktion}
\label{fig:traegheitsfusion}
\end{figure}
\end{frame}

\subsection{Erbrütung von Tritium}
\begin{frame}{\subsecname \ aus Lithium-6}{\secname}
\pftn{https://de.wikipedia.org/wiki/Blanket}
\begin{itemize}
\item zu 90 \% 
\item \(^{6}{\mathrm  {Li}}+{\mathrm  {n}}\ \rightarrow \ ^{4}{\mathrm  {He}}+{}^{3}{\mathrm  {H}}+4{,}8\;{\mathrm  {MeV}}\)
\item Hohe Energieausbeute
\end{itemize}
\end{frame}

\begin{frame}{\subsecname \ aus Lithium-7}{\secname}
\pftn{https://de.wikipedia.org/wiki/Blanket}
\begin{itemize}
\item zu 10 \%
\item \(^{7}{\mathrm  {Li}}+{\mathrm  {n}}\ \rightarrow \ ^{4}{\mathrm  {He}}+{}^{3}{\mathrm  {H}}+{\mathrm  {n}}'-2{,}5\;{\mathrm  {MeV}}\)
\item Neutron nicht verbraucht
\item Hohe Energieschwelle
\end{itemize}
\end{frame}

\subsection{Sicherheit der Reaktoren}
\begin{frame}{\subsecname: \ Vorteile}{\secname}
\pftn{http://www.scinexx.de/dossier-detail-134-12.html}
\begin{itemize}
\item Keine Kettenreaktionen --> kein Super-GAU
\item Ohne Kühlung kommt Fusionsreaktion nicht zustande
\item Halbwertszeit von Tritium nur 12,3 Jahre
\end{itemize}
\end{frame}

\begin{frame}{\subsecname: \ Nachteile}{\secname}
\pftn{http://www.scinexx.de/dossier-detail-134-12.html}
\begin{itemize}
\item Tritium ist leicht --> kann durch Lecks entweichen
\item Beta-Strahlung kann beim einatmen Zellen schaden
\item Bauteile können durch Neutronenstrahlung aktiviert werden
\item 10 \% der Bauteile für 100 Jahre verstrahlt
\end{itemize}
\end{frame}

\begin{frame}
\titlepage
\end{frame}
\end{document}


%%  Energie aus Masse E=mc²
%% Geschichtlicher Überblick der Forschung
%% Unterwelchen Bedingungen funktioniert die Fusion (Hitze, Druck)
%% Fusion in der Sonne. In welchen Zonen und in welchen Lebenzzyklen finden welche Fusionsreaktionen statt.
%% Wasserstoffbombe: Erste Fusion auf der Erde.
%% Zwei Prinzipien der Fusionsreaktoren: magnetische Fusion (Tokamak / Stellarator), Trägheitsfusion
%% Aufbau eines Reaktors
%% Erbrütung von Tritium
%% Sicherheit von Fusionsreaktoren 
