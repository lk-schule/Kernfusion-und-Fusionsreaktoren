\documentclass[10pt,a4paper, ngerman]{beamer}

\include{beamer}

\AtBeginSection{\frame{\frametitle{Gliederung}\tableofcontents[currentsection]}}

\setbeamercovered{transparent}
\author{Luca Kiebel}
\title{Kernfusion und Fusionsreaktoren}
%\subtitle{subtitle}
\date{\today}
\institute[HBBK]{Hans-Böckler-Berufskolleg}
\setlength{\itemsep}{10pt}
\begin{document}
\begin{frame}
\titlepage
\end{frame}

\section[Energie aus Masse]{Äquivalenz von Masse und Energie}
\begin{frame}{\secname}{\subsecname}
\begin{itemize}[<+->]
\item \(E=mc^2\)
\item E => Ruheenergie (ohne Bewegung)
\end{itemize}
\end{frame}

\begin{frame}{\secname}{\(E=mc^2\) vs. \(m*g*h\)}
\begin{exampleblock}{Beispiel 1: Kugelschreiber im freien Fall}
	\textasciitilde14g Gewicht \\
	0.22 J
\end{exampleblock}
\pause
\begin{exampleblock}{Beispiel 2: Little Boy} %6. August 1945 Hiroshima
	\alt<3,4>{> 70 Kugelschreiber}{< 1 kg} genzündetes Reaktionsmaterial\ftn{1}{https://de.wikipedia.org/wiki/Little\_Boy} \\
	\(5.4*10^{16}\) J 
\end{exampleblock}
\pause[4]
\begin{exampleblock}{Beispiel 3: Katze}
	\textasciitilde5kg Gewicht \\
	\textasciitilde\(4.2*10^{17}\) J\ftn{2}{https://youtu.be/t-O-Qdh7VvQ?t=10}
\end{exampleblock}
\end{frame}


\section{Kernfusion}
\subsection{Geschichte der Forschung}
\begin{frame}{\subsecname}{\secname}
\begin{enumerate}
\item 1917: Erste Kernreaktion (Rutherford)\ftn{1}{http://web.lemoyne.edu/{\textasciitilde}giunta/rutherford.html}
\item 1920: Fusionsreaktion mögliche Energiequelle von Sternen\ftn{2}{Hans Bethe: \textit{Energy Production in Stars}{,} Phys. Rev. 55{,} 1939{,} S. 434–456}
\item 1934: Erste Fusionsreaktion im Labor\ftn{3}{M.L.E. Oliphant{,} Lord Rutherford: \textit{Transmutation effects Observed with Heavy Hydrogen}{,} Rev. 144{,} 1934{,} S. 692}
\item ab 1945: Erforschung der Nutzung von FR in Atombomben
\item 1952: Zündung der ersten Wasserstoffbombe\ftn{4}{http://nuclearweaponarchive.org/Usa/Tests/Ivy.html}
\item 1991: Erste kontrollierte Kernfusion zur Energiegewinnung\ftn{5}{P-H Rebut: \textit{The JET preliminary tritium experiment}{,} Rev. 34{,} 1992}
\end{enumerate}
\end{frame}

\subsection{Bedingungen für die Fusion}
\begin{frame}{\subsecname}{\secname}

\end{frame}


\subsection{Fusion in der Sonne}
\begin{frame}{\subsecname}{\secname}

\end{frame}

\subsection{Wasserstoffbombe}
\begin{frame}{\subsecname}{\secname}

\end{frame}



\section{Fusionsreaktor}
\subsection[Magnetische Fusion]{Magnetische Fusion (Tokamak / Stellerator)}
\begin{frame}{\subsecname}{\secname}

\end{frame}

\subsection{Trägheitsfusion}
\begin{frame}{\subsecname}{\secname}

\end{frame}

\subsection{Aufbau eines Reaktors}
\begin{frame}{\subsecname}{\secname}

\end{frame}

\subsection{Erbrütung von Tritium}
\begin{frame}{\subsecname}{\secname}

\end{frame}

\subsection{Sicherheit der Reaktoren}
\begin{frame}{\subsecname}{\secname}

\end{frame}



\begin{frame}
\titlepage
\end{frame}
\end{document}


%%  Energie aus Masse E=mc²
%% Geschichtlicher Überblick der Forschung
%% Unterwelchen Bedingungen funktioniert die Fusion (Hitze, Druck)
%% Fusion in der Sonne. In welchen Zonen und in welchen Lebenzzyklen finden welche Fusionsreaktionen statt.
%% Wasserstoffbombe: Erste Fusion auf der Erde.
%% Zwei Prinzipien der Fusionsreaktoren: magnetische Fusion (Tokamak / Stellarator), Trägheitsfusion
%% Aufbau eines Reaktors
%% Erbrütung von Tritium
%% Sicherheit von Fusionsreaktoren 
